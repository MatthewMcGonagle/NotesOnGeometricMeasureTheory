\subsection{The Product of Covectors (Federer 1.4.2)}

Here we discuss Federer's choice of definition for the product of two covectors \(\psi  \wedge \omega\) for \(\psi \in \Lambda^m V, \omega \in \Lambda^n V\). There are other choices for
forming a product that at first glance may seem natural. To differentiate such a product from \(\psi \wedge \omega\), we will denote it by \(p(\psi, \omega)\). Recall that
we can think of \(\psi\) as a linear function \(\psi: \Lambda_m V \to \mathbb R\) and similarly for \(\omega\). So to define a product \(p(\psi, \omega)\) of
\(\psi\) and \(\omega\), we wish to define a linear function \(p(\psi, \omega): \Lambda^{m+n} V \to \mathbb R\). Initially, the following choice for \(p\) may seem natural,
i.e. the linear map generated by:
\begin{equation}
p(\psi, \omega)(v_1 \wedge ... \wedge v_{m+n}) \coloneqq
    \sum\limits_{\sigma \in S_{m+n}} \ind(\sigma) \psi\left(v_{\sigma(1)} \wedge ... \wedge v_{\sigma(m)} \right) \omega\left(v_{\sigma(m+1)} \wedge ... \wedge v_{\sigma(m + n)}\right),
\end{equation}
where \(S_{m+n}\) is the permuation group of \(m + n\) items.

However, this choice is missing a nice property that is satisfied by \(\psi \wedge \omega\). Consider the case that \(V\) is three-dimensional with basis \(\{e_1, e_2, e_3\}\) and
consider the dual basis \(\{\epsilon_1, \epsilon_2, \epsilon_3\}\) of \(\Lambda^1 V\). For \(\Lambda_2 V\) we have the basis
\(\{e_{12} = e_1 \wedge e_2, e_{13} = e_1 \wedge e_3, e_{23} = e_2 \wedge e_3\}\) with
dual basis \(\{\epsilon_{12}, \epsilon_{13}, \epsilon_{23}\}\) of \(\Lambda^2 V\). So, for example, \(\epsilon_{12}(e_1 \wedge e_2) = 1\). Similarly we have a basis
\(\{e_{123} = e_1 \wedge e_2 \wedge e_3\}\) for \(\Lambda_3 V\) and dual basis \(\{\epsilon_{123}\}\) of \(\Lambda^3 V\). The nice property is that 
\(\epsilon_{123} = \epsilon_1 \wedge \epsilon_2 \wedge \epsilon_3 = \epsilon_1 \wedge \epsilon_{23}\).

However, the product \(p\) does not enjoy this property. Note that 
\begin{align}
p(\epsilon_1, \epsilon_{23})( e_1\wedge e_{23} ) & =  \epsilon_1(e_1) \epsilon_{23}(e_2 \wedge e_3) - \epsilon(e_1) \epsilon_{23}(e_3 \wedge e_2), \\
    & = 1 - (-1) \\
    & = 2.
\end{align}
So \(p(\psi, \omega)\) is in fact not a natural choice of product.

Note that, as Federer shows, \(\psi \wedge \omega \left(v_1 \wedge ... \wedge v_{m+n}\right)\) can be expressed as a sum over the shuffle permutations of \(m+n\) items instead of the full
permutation group. However, Federer chooses instead to define \(\psi \wedge \omega\) by using a natural map \(\Lambda^* V \to \Lambda^* V \otimes \Lambda^* V\) where the latter is the
anti-commutative \(\otimes\). This map effectively
picks out choices of partitions of the elements of \((v_1, ..., v_{m+n})\) into those that will go to \(\psi\) (i.e. the terms \(v_i \otimes 1\)) and those that will go to \(\omega\)
 (i.e. those terms \(1 \otimes v_i\)); furthermore, it preserves the order and
picks out the correct value of sign that would be needed to move the elements around to make the equivalent shuffle partition.
