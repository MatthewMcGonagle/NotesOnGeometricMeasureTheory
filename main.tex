\documentclass{article}
\usepackage{mathtools}
\usepackage{amsthm}

\newtheorem{lemma}{Lemma}
\newtheorem{proposition}{Proposition}

\begin{document}
\title{Notes on Geometric Measure Theory}
\author{Matthew McGonagle}
\maketitle
\section{Introduction}
The purpose of these notes is mostly to expand discussions or explain details of arguments
in Federer's Geometric Measure Theory. So the layout of these notes will reflect the relevant location
of the topic from Federer.

Note that we also use Federer's definition of a measure, which in standard treatments would be called
an outer measure.

\section{Measure Theory (Federer Chapter 2)}
\subsection{Suslin Sets}

Let \(\mathcal P = \{1, 2, 3, ...\}\) denote the set of positive integers and let
\(\mathcal N = {\mathcal P}^{\mathcal P}\) be the set of sequences of positive integers.

The topology on \(\mathcal N\) is generated by the following open sets: for any \(x \in \mathcal N\) and
\(i_0 \geq 1\), the open set
\(U = \{(x_1, ..., x_{i_0 - 1}, y_{i_0}, y_{i_0 + 1}, ...) \mid y_i \in \mathcal P\}\) for some \(i_0 \geq 1\).
This coincides with the topology of the metric used by Federer.

\subsubsection{A Compact Subspace of \(\mathcal N\) (Federer 2.2.12)}

Given a sequence of positive integers \(m_i\), consider the sub-space
\(K \coloneqq \{n \in \mathcal N \mid n_i \leq m_i\} \subset \mathcal N\). We show that \(K\) is a compact
sub-space of \(\mathcal N\). First note that if \(m_i > 1\) for only a finitely many \(i\), then \(K\) is a finite
set and must be compact. Therefore, we need only consider the case that \(m_i > 1\) for infinitely many  \(i\).

To accomplish this, we use the lexiographic ordering on \(\mathcal N\). That is, \(x < y\) for
\(x, y \in \mathcal N\) if \(x_i < y_i\) for the smallest \(i\) such that \(x_i \neq y_i\).

First, a lemma. 
\begin{lemma}
Given bounded set \(S \subset \mathcal N\), we have that \(\sup S\) exists, i.e. that \(S\) has least
upper bound.
\end{lemma}
\begin{proof}
We recursively construct \(m \in \mathcal N\) such that \(s \leq m\) for all \(s \in S\). Since, \(S\) is
bounded, we have that the set of positive integers \(\{s_1 \mid s \in S\}\) is bounded. We let \(m_1\) be this
maximum. 

Now, given \(j\), suppose we have constructed \(m_i\) for \(i < j\) such that there exists \(s \in S\) with
\(s_i = m_i\) for \(i < j\) and for any \(s \in S\) we have that \(s_i \leq m_i\) for \(i < j\). 

Now, if the set of positive integers
\(S_j \coloneqq \{s_j \mid s \in S \text{ and } s_i = m_i \text{ for } i < j\}\) is unbounded, then we may choose 
\(m = (m_1,..., m_{j-2}, 1 + m_{j-1}, 1, 1, ...)\) and we see that \(m\) is a least upper bound of \(S\).
So consider the case that this set is bounded. Then we let \(m_j\) be its maximum.

We have recursively constructed our \(m_j\). We need only consider the case that the sets \(S_j\) are bounded
for each step of our recursion. Clearly \(m\) is an upper bound for \(S\). Furthermore, consider any \(x < m\)
such that \(i\) is the first component where \(x_i < m_i\). Then by construction, there is an \(s \in S_i\) such
that \(s > x\). Hence \(m\) is a least upper bound. 
\end{proof}

Now consider an open covering \(U_\alpha\) of \(K\) and the set \(S\) of points \(s \in \mathcal N\) such
that \(s \leq m\) and the set \(K_s \coloneqq \{k \mid k \in K \text{ and } k \leq s\}\) is covered by a
finite sub-collection of \(U_\alpha\). Since \(1 \coloneqq (1, 1, 1, ...) = \min K\), we see that \(1 \in S\).
Furthermore, by construction \(S\) is bounded. 

Next, note that for some \(i_0\), all points in \(K\) of the form \((1, 1, ..., 1, n_{i_0}, n_{i_0 + 1}, ...)\)
are in an open set \(U_1\) covering \(1\). Since \(m_i > 1\) for infinitely many \(i\), this means that there
exists \(k_0 \in K \cap U_1\) such that all \(K_{k_0} \subset U_1\). Therefore \(M \coloneqq \sup S > 1\).

Given any \(x \in \mathcal N\) that is an upper bound for \(S\), we have that \(y \in \mathcal N\) defined by
\(y_i \coloneqq \min \{x_i, m_i\}\) is also an upper bound for \(S\). Therefore \(M_i \leq m_i\); note that
this is a stronger condition than \(M \leq m\). 

We wish to show that \(M \in S\). Since \(M_i \leq m_i\), we have that \(M \in K\). So let \(U_M\) be an
open set covering \(M\). So we know that there exists \(i_0\) such that all
points in \(K\) of the form \((M_1, M_2, ..., M_{i_0}, n_{i_0+1}, n_{i_0 + 2}, ...)\) for
general \(n_i\) are in \(U_M\).  

Consider the case that \(M_{j_0} > 1\) for some \(j_0 > i_0\). Then there exists a point \(k \in K\) such that
\(k < M\) and \(k \in U_M\), namely \(k = (M_1, ..., M_{i_0}, 1, 1, 1, ...)\). Note that all \(x \in K\)
satisfying \(k \leq x \leq M\) are also in \(U_M\). By the definition of \(M\), we
have that \(K_k\) is covered by a finite sub-collection \(U_\alpha\). When combined with \(U_M\)
we see that \(K_M\) is also covered by a finite sub-collection and so \(M \in S\). 

Consider the final case that \(M_j = 1\) for all \(j > i_0\), i.e. \(M = (M_1, ..., M_{i_0}, 1, 1, 1, ...)\).
We know that \(M > 1\) so there must exist a greatest \(j_0 \leq i_0\) such that \(M_{j_0} > 1\). Consider
the point
\(x = (M_1, ..., M_{j_0}-1, m_{j_0 + 1}, m_{j_0 + 2}, ...) \in K\). We have that \(x < M\) and that there are
no points \(y \in K\) such that \(x < y < M\). We may cover \(K_x\) with a finite sub-collection of the
\(U_\alpha\), and then we combined with \(U_M\) we have a finite sub-collection covering \(K_M\). Therefore
\(M \in S\). 

We prove by contradiction that \(M = m\). Consider if \(M < m\). Similar to above, consider an open set
\(U_M\) covering \(M\), and consider the case that there is a \(j_0 > i_0\) such that \(M_{j_0} < m_{j_0}\).
Let \(x = (M_1, ..., M_{j_0 - 1}, m_{j_0}, 1, 1, 1, ...)\); note that all
\(y \in K\) such that \(M < y \leq x\) are also covered by \(U_M\). Therefore \(K_x\) will also be
covered by a finite sub-collection of \(U_\alpha\), which contradicts the definition of \(M\).

Consider the final case that \(M_i = m_i\) for all \(i > i_0\) but that there exists a \(j_0 \leq i_0\) with
\(M_{j_0} < m_{j_0}\). We may take \(j_0\) to be the largest such number. Let 
\(x = (M_1, ..., M_{j_0 - 1}, M_{j_0} + 1, 1, 1, 1, ...) \in K\). We note that there are no \(y \in K\) such
that \(M < y < x\). Since \(M \in S\), we may cover \(K_M\) with a finite sub-collection of \(U_\alpha\); this
combined with a cover \(U_x\) of \(x\) gives a finite sub-cover for \(K_x\), which contradicts the
definition of \(M\). 

Therefore we have \(M = m\) and \(K\) may be covered by a finite sub-collection of \(U_\alpha\).

\subsection{An Identity for the Upper Integral based on Caratheodory's Construction (Federer 2.10.24 (1))} 
Federer shows that the quantity \(\lambda_\delta(f)\) based on Caratheodory's construction matches the upper
Lebesgue Integral \(\int^* f d\psi\) in certain cases. We will prove the following
statement from the case of (Federer 2.10.24 (1)),
\begin{proposition}
In the case of (Federer 2.10.24 (1)), there exists an increasing sequence of
Borel sets \(B_i\) and a sequence of \(\epsilon_i > 0\) such that
\begin{equation}
\psi\left(\{x \mid f(x) > 0 \} \setminus \bigcup_i B_i \right) = 0,
\end{equation}
and \(\psi(B_i \cap S) \leq (1 + i^{-1}) \zeta(S)\) whenever \(S \in F\), \(B_i \cap S \neq \emptyset\),
and \(\text{diam }S \leq \epsilon_i\).
\end{proposition}

Now we prove the proposition.
\begin{proof}
It is clear that we may write \(\{x \mid f(x) > 0\} = \bigcup_j A_j\) for an increasing sequence of sets
\(A_j\) with \(\psi(A_j) < \infty\). Given \(i\),  consider the set of
\(C_{ij} = \{x \mid \psi(A_i \cap S) \leq (1 + i^{-1}) \zeta(S) \text{ for } x \in S
    \text{ and diam } S \leq 1/j\}\). Note that \(C_{ij}\) is increasing in \(j\) and that 
\(\psi\left( A_i \setminus \bigcup_{j} C_{ij} \right) = 0\) by (Federer 2.10.18 (3)). Therefore, we may
find a set \(\tilde C_i\) and \(\epsilon_i > 0\) such that for all \(x \in \tilde C_i\) we have that
\(\psi(A_i \cap S) \leq (1 + i^{-1}) \zeta(S)\)
whenever \(x \in S\) and \(\text{diam } S < \epsilon_i\)
; furthermore \(\psi(A_i \setminus \tilde C_i) \leq 2^{-i - 1}\). In particular,
\(\psi(\tilde C_i \cap S) \leq (1 + i^{-1}) \zeta(S)\) whenever \(S \cap \tilde C_i \neq \emptyset\) and
\(\text{diam } S \leq \epsilon_i\).

From the Borel regularity of \(\psi\) we may find Borel sets \(B_i\) such that \(\tilde C_i \subset B_i\) and
\(\psi(B_i) = \psi(\tilde C_i)\); note that \(B_i\) must then be a \(\psi\)-hull of \(\tilde C_i\).
In particular, since
each \(S \in F\) is closed and hence \(\psi\) measurable, we have that 
\(\psi(B_i \cap S) \leq (1 + i^{-1}) \zeta(S)\) whenever \(S \cap \tilde C_i \neq \emptyset\) and
\(\text{ diam } S \leq \epsilon_i\). Since each \(S \in F\) is measurable (since closed) and
\(B_i\) is a \(\psi\)-hull of \(\tilde C_i\), we have that \(\psi(B_i \cap S) = \psi(\tilde C_i \cap S)\) for all
\(S \in F\). Furthermore, if \(S \cap B_i \neq \emptyset\) but
\(S \cap \tilde C_i = \emptyset\), then \(\psi(B_i \cap S) = 0\). So we we have
that \(\psi(B_i \cap S) \leq (1 + i^{-1}) \zeta(S)\) whenever \(S \cap B_i \neq \emptyset\) and
\(\text{ diam } S \leq \epsilon_i\).

To form an increasing sequence of Borel sets, we set \(\tilde B_i = \bigcap_{j \geq i} B_j\). We see that
\(\psi(A_i \setminus \tilde B_i) \leq \sum_{j \geq i} \psi(A_i \setminus B_j) \leq 2^{-i}\). Furthermore,
we have that \(\psi(\tilde B_i \cap S) \leq (1 + i^{-1}) \zeta(S)\) whenever
\(S \cap \tilde B_i \neq \emptyset\) and
\(\text{ diam } S \leq \epsilon_i\).

\end{proof}


\end{document}
