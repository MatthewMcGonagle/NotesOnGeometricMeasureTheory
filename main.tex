\documentclass{article}
\usepackage{mathtools}
\usepackage{amsthm, amsfonts}

\newtheorem{lemma}{Lemma}
\newtheorem{proposition}{Proposition}
\DeclareMathOperator{\ind}{ind}

\begin{document}
\title{Notes on Geometric Measure Theory}
\author{Matthew McGonagle}
\maketitle
\section{Introduction}
The purpose of these notes is mostly to expand discussions or explain details of arguments
in Federer's Geometric Measure Theory. So the layout of these notes will reflect the relevant location
of the topic from Federer.

Note that we also use Federer's definition of a measure, which in standard treatments would be called
an outer measure.

\section{Grassmann Algebra (Federer Chapter 1)}
\subsection{The Product of Covectors (Federer 1.4.2)}

Here we discuss Federer's choice of definition for the product of two covectors \(\psi  \wedge \omega\) for \(\psi \in \Lambda^m V, \omega \in \Lambda^n V\). There are other choices for
forming a product that at first glance may seem natural. To differentiate such a product from \(\psi \wedge \omega\), we will denote it by \(p(\psi, \omega)\). Recall that
we can think of \(\psi\) as a linear function \(\psi: \Lambda_m V \to \mathbb R\) and similarly for \(\omega\). So to define a product \(p(\psi, \omega)\) of
\(\psi\) and \(\omega\), we wish to define a linear function \(p(\psi, \omega): \Lambda^{m+n} V \to \mathbb R\). Initially, the following choice for \(p\) may seem natural,
i.e. the linear map generated by:
\begin{equation}
p(\psi, \omega)(v_1 \wedge ... \wedge v_{m+n}) \coloneqq
    \sum\limits_{\sigma \in S_{m+n}} \ind(\sigma) \psi\left(v_{\sigma(1)} \wedge ... \wedge v_{\sigma(m)} \right) \omega\left(v_{\sigma(m+1)} \wedge ... \wedge v_{\sigma(m + n)}\right),
\end{equation}
where \(S_{m+n}\) is the permuation group of \(m + n\) items.

However, this choice is missing a nice property that is satisfied by \(\psi \wedge \omega\). Consider the case that \(V\) is three-dimensional with basis \(\{e_1, e_2, e_3\}\) and
consider the dual basis \(\{\epsilon_1, \epsilon_2, \epsilon_3\}\) of \(\Lambda^1 V\). For \(\Lambda_2 V\) we have the basis
\(\{e_{12} = e_1 \wedge e_2, e_{13} = e_1 \wedge e_3, e_{23} = e_2 \wedge e_3\}\) with
dual basis \(\{\epsilon_{12}, \epsilon_{13}, \epsilon_{23}\}\) of \(\Lambda^2 V\). So, for example, \(\epsilon_{12}(e_1 \wedge e_2) = 1\). Similarly we have a basis
\(\{e_{123} = e_1 \wedge e_2 \wedge e_3\}\) for \(\Lambda_3 V\) and dual basis \(\{\epsilon_{123}\}\) of \(\Lambda^3 V\). The nice property is that 
\(\epsilon_{123} = \epsilon_1 \wedge \epsilon_2 \wedge \epsilon_3 = \epsilon_1 \wedge \epsilon_{23}\).

However, the product \(p\) does not enjoy this property. Note that 
\begin{align}
p(\epsilon_1, \epsilon_{23})( e_1\wedge e_{23} ) & =  \epsilon_1(e_1) \epsilon_{23}(e_2 \wedge e_3) - \epsilon(e_1) \epsilon_{23}(e_3 \wedge e_2), \\
    & = 1 - (-1) \\
    & = 2.
\end{align}
So \(p(\psi, \omega)\) is in fact not a natural choice of product.

Note that, as Federer shows, \(\psi \wedge \omega \left(v_1 \wedge ... \wedge v_{m+n}\right)\) can be expressed as a sum over the shuffle permutations of \(m+n\) items instead of the full
permutation group. However, Federer chooses instead to define \(\psi \wedge \omega\) by using a natural map \(\Lambda^* V \to \Lambda^* V \otimes \Lambda^* V\) where the latter is the
anti-commutative \(\otimes\). This map effectively
picks out choices of partitions of the elements of \((v_1, ..., v_{m+n})\) into those that will go to \(\psi\) (i.e. the terms \(v_i \otimes 1\)) and those that will go to \(\omega\)
 (i.e. those terms \(1 \otimes v_i\)); furthermore, it preserves the order and
picks out the correct value of sign that would be needed to move the elements around to make the equivalent shuffle partition.


\section{Measure Theory (Federer Chapter 2)}
\subsection{Suslin Sets}

\subsubsection{A Compact Subspace of \(\mathcal N\) (Federer 2.2.12)}

Given a sequence of positive integers \(m_i\), consider the sub-space
\(K \coloneqq \{n \in \mathcal N \mid n_i \leq m_i\} \subset \mathcal N\). We show that \(K\) is a compact
sub-space of \(\mathcal N\). 

\subsection{An Identity for the Upper Integral based on Caratheodory's Construction (Federer 2.10.24 (1))} 
Federer shows that the quantity \(\lambda_\delta(f)\) based on Caratheodory's construction matches the upper
Lebesgue Integral \(\int^* f d\psi\) in certain cases. We will prove the following
statement from the case of (Federer 2.10.24 (1)),
\begin{proposition}
In the case of (Federer 2.10.24 (1)), there exists an increasing sequence of
Borel sets \(B_i\) and a sequence of \(\epsilon_i > 0\) such that
\begin{equation}
\psi\left(\{x \mid f(x) > 0 \} \setminus \bigcup_i B_i \right) = 0,
\end{equation}
and \(\psi(B_i \cap S) \leq (1 + i^{-1}) \zeta(S)\) whenever \(S \in F\), \(B_i \cap S \neq \emptyset\),
and \(\text{diam }S \leq \epsilon_i\).
\end{proposition}

Now we prove the proposition.
\begin{proof}
It is clear that we may write \(\{x \mid f(x) > 0\} = \bigcup_j A_j\) for an increasing sequence of sets
\(A_j\) with \(\psi(A_j) < \infty\). Given \(i\),  consider the set of
\(C_{ij} = \{x \mid \psi(A_i \cap S) \leq (1 + i^{-1}) \zeta(S) \text{ for } x \in S
    \text{ and diam } S \leq 1/j\}\). Note that \(C_{ij}\) is increasing in \(j\) and that 
\(\psi\left( A_i \setminus \bigcup_{j} C_{ij} \right) = 0\) by (Federer 2.10.18 (3)). Therefore, we may
find a set \(\tilde C_i\) and \(\epsilon_i > 0\) such that for all \(x \in \tilde C_i\) we have that
\(\psi(A_i \cap S) \leq (1 + i^{-1}) \zeta(S)\)
whenever \(x \in S\) and \(\text{diam } S < \epsilon_i\)
; furthermore \(\psi(A_i \setminus \tilde C_i) \leq 2^{-i - 1}\). In particular,
\(\psi(\tilde C_i \cap S) \leq (1 + i^{-1}) \zeta(S)\) whenever \(S \cap \tilde C_i \neq \emptyset\) and
\(\text{diam } S \leq \epsilon_i\).

From the Borel regularity of \(\psi\) we may find Borel sets \(B_i\) such that \(\tilde C_i \subset B_i\) and
\(\psi(B_i) = \psi(\tilde C_i)\); note that \(B_i\) must then be a \(\psi\)-hull of \(\tilde C_i\).
In particular, since
each \(S \in F\) is closed and hence \(\psi\) measurable, we have that 
\(\psi(B_i \cap S) \leq (1 + i^{-1}) \zeta(S)\) whenever \(S \cap \tilde C_i \neq \emptyset\) and
\(\text{ diam } S \leq \epsilon_i\). Since each \(S \in F\) is measurable (since closed) and
\(B_i\) is a \(\psi\)-hull of \(\tilde C_i\), we have that \(\psi(B_i \cap S) = \psi(\tilde C_i \cap S)\) for all
\(S \in F\). Furthermore, if \(S \cap B_i \neq \emptyset\) but
\(S \cap \tilde C_i = \emptyset\), then \(\psi(B_i \cap S) = 0\). So we we have
that \(\psi(B_i \cap S) \leq (1 + i^{-1}) \zeta(S)\) whenever \(S \cap B_i \neq \emptyset\) and
\(\text{ diam } S \leq \epsilon_i\).

To form an increasing sequence of Borel sets, we set \(\tilde B_i = \bigcap_{j \geq i} B_j\). We see that
\(\psi(A_i \setminus \tilde B_i) \leq \sum_{j \geq i} \psi(A_i \setminus B_j) \leq 2^{-i}\). Furthermore,
since \(\tilde B_i \subset B_i\), we have that \(\psi(\tilde B_i \cap S) \leq (1 + i^{-1}) \zeta(S)\) whenever
\(S \cap \tilde B_i \neq \emptyset\) and
\(\text{ diam } S \leq \epsilon_i\).

Next, note that \(\psi\left(A_i \setminus \bigcup_j \tilde B_j\right) \leq \psi(A_i \setminus \tilde B_j) \leq
2^{-j}\)
for any \(j \geq i\) (recall that \(A_i\) is increasing). Therefore,
\(\psi\left(A_i \setminus \bigcup_j \tilde B_j\right) = 0\), and we get that
\(\psi\left( \{ x \mid f(x) > 0 \} \setminus \bigcup_j \tilde B_j \right) = 0\).
\end{proof}

\subsection{Collections of Closed Intervals that are Special for a Given Interval (Federer 2.10.28)}

Here we discuss in more detail the proof of item (3) of Federer 2.10.28. In particular, we show that 
\(\psi(A \cap T) = \zeta(T)\) when \(T \in \bigcup_j H_j\).

\begin{lemma}
Consider a covering of \(A \cap T\) by compact intervals \(R_i \in F\). Given \(\epsilon > 0\), we may
cover \(A \cap T\) by a countable collection \(S_i \in \bigcup_j H_j\) such that each \(S_i \subset T\) and
\(\sum_i \zeta(S_i) \leq \epsilon + \sum_i \zeta(R_i)\).
\end{lemma}
\begin{proof}
Consider any compact interval \(R\). Given \(j\), consider the
\(S \in H_j\) such that \(S \subset T\) and \(S \cap R \neq \emptyset\). Since \(R\) is a compact interval
and the \(S\) are compact disjoint intervals, there are at most
two different \(S \in H_j\) such that \(S \cap R \neq \emptyset\) and \(S\) is not a subset of \(R\); these
are any \(S\) that contain the endpoints of \(R\) but are not subsets of \(R\). 

Since, \(\lim\limits_{j\to \infty}\sup\limits_{S\in H_j} \text{diam } S = 0\), given \(\epsilon > 0\) and
\(k\) we may find \(j\) and \(S_i \in H_j\) satisfying \(S_i \subset T\),
\(T \cap A \cap R \subset \bigcup_i S_i\), and 
\(\sum_i \zeta(S_i) \leq \epsilon 2^{-1-k} + \zeta(R)\).

Now given a collection of compact intervals \(R_i\) satisfying \(A \cap T \subset \bigcup_i R_i\), do the above
to find \(S_{ij} \subset T\) such that \(A \cap T \cap R_i \subset \bigcup_j S_{ij}\) and
\(\sum_j \zeta(S_{ij}) \leq \epsilon 2^{-1 - i} + \zeta(R_i)\). Then we find that the countable collection
\(\bigcup_i \bigcup_j \{S_{ij}\}\) satsifies
\(\sum_i \zeta(S_i) \leq \epsilon + \sum_i \zeta(R_i)\). 
\end{proof}

From the above lemma, when considering \(\psi(A\cap T)\), we need only consider covering \(A \cap T\) by
countable collections of \(S_i \subset T\).

\begin{lemma}
Given a countable collection \(S_i \subset T\) covering \(T \cap A\), there is a finite sub-covering of
\(T \cap A\).
\end{lemma}
\begin{proof}
We will be taking advantage of the fact that \(S \in H_j\) are disjoint.
\end{proof}


\section{Rectifiability (Federer Chapter 3)}
\subsection{Correction of Federer 3.2.16}

When describing the \((\phi, m)\) approximate differential of \(f\) in Federer 3.2.16, Federer states that \(f\) determines the restriction of \(Dg(a)\) to \(\text{Tan}[\{x : f(x) = g(x)\}, a]\),
but this is not true; however it is true that \(f\) determines the restriction of \(Dg(a)\) to \(\text{Tan}^m(\phi, a)\). Let's look at a counter-example to Federer's claim.

Consider \(\mathbb R^2\), \(\phi = \mathcal H ^1 |_{\{y = 0\}} \), 
\begin{equation}
f(x, y) = \begin{cases}
    x, & y = 0, \\
    x + y, & x = y^2, \\
    0, & x= 0, \\
    -1, & \text{otherwise},
\end{cases}
\end{equation}
\(g_1(x, y) = x\), and \(g_2(x, y) = x + y\). Note that \(\{f(x) = g_1(x) \} = \{y = 0 \} \cup \{x = 0\} \cup \{x = -1\}\) and \(\{f(x) = g_2(x)\} = \{y= 0\} \cup \{x= y^2\} \cup \{x + y = -1\}\).
So we see that \(\text{Tan}[\{f(x) = g_1(x)\}, 0] = \text{Tan}[\{f(x) = g_1(x)\}, 0] = \{y = 0\} \cup \{x = 0\}\). Also note that
\(\text{Tan}^1 (\phi, 0) = \{y = 0\}\) and that \(\phi\left(\{f(x) \neq g_1(x)\}\right) = \phi\left(\{f(x) \neq g_2(x)\}\right) = 0\) means that the densities
\(\Theta^1(\{f(x) \neq g_1(x)\}, 0) = \Theta^1(\{f(x) \neq g_2(x)\}, 0) = 0\).
Therefore, we see that \(g_1\) and \(g_2\) both meet the conditions for determining the approximate derivative of \(f\), and that\(Dg_1(0)\) and \(Dg_2(0)\) both agree on \(\text{Tan}^1 (\phi, 0)\);
however they do not agree on \(\text{Tan}[\{f(x) = g_1(x)\}, 0]\) and \(\text{Tan}[\{f(x) = g_2(x)\}, 0]\).

\subsection{Some Remarks on Fereder's Theorem 3.2.5}
For Federer's Theoerm 3.2.5, the combination of its conditions, its proof, and Federer's short-hand use of integral notation combine to make the theorem a little confusing. Here are some remarks
to help clarify this issue.

First, recall that when Federer uses \(\int f d\phi\), it includes being used as a short-hand notation for the fact that \(f\) is a \(\phi\)-measurable function and that the integral exists
(but not necessarily finite). Furthermore, \(\int_A f d\\phi\)
includes the shorthand that \(f\) restricted to \(A\) is measurable and exists.

Conditions one (i.e. \(g\) is measurable) and two (i.e. \(N(f|A, y) < \infty\) for \(\mathcal H^m\) almost \(y\)) of the theorem should be interpreted as conditions applying to the right hand side
integral of the theorem:
\begin{equation}
\int_A g\left(f(x)\right) J_m f(x) d \mathcal L^mx = \int_{\mathbb R^n} g(y) N(f|A, y) d \mathcal H^m y.
\end{equation}
So both include the assumption that the integrand of the left hand side is measurable. This fact is redundant for condition one, but it is important to understand condition two and its proof. 

An important  point in the proof for condition one is that we must show that the left hand side integrand restricted to \(A\) is a measurable function (note that \(g(f(x))\) is not necessarily
a measurable function by itself). Following the convention of Federer's condition three,
we use \(\alpha\) to denote the characteristic function of \(A\), and for simplicity we use \(h = \alpha g\left(f(x)\right) J_m f(x)\) denote the left integrand. So we must show that \(h\) is
a measurable function. 

As in Federer's proof of condition one, we fist consider the case that \(g\) is the characteristic function of an \(\mathcal H^m\) measurable subset \(T\) of \(f(A)\). Since \(f\) is
Lipschitz, we know that \(f^{-1}(B)\) is measurable for every \(B\). Using this and Federer's comment about decomposing \(T\) into a Borel set and a set with zero \(\mathcal H^m\) measure,
we need only consider the case that \(\mathcal H^m(T) = 0\). 

From the Borel regularity of \(\mathcal H^m\), there exists a Borel set \(B\) such that \(T \subset B\) and \(H^m(B) = H^m(T) = 0\). Let \(\tilde g\) be the characteristic function of \(B\), and
we let \(\tilde h = \alpha \tilde g\left(f(x)\right) J_m f(x)\). From Federer Theorem 3.2.3 (2), we know that \(\int \tilde h d \mathcal L^m = 0\). Therefore, we know that 
\(\mathcal L^m \left( \{ \tilde h > 1/k \} \right) = 0\) for any positive integer \(k\). Hence \(\mathcal L^m \left( \{ \tilde h > 0 \} \right) = 0\). Since \(h \leq \tilde h\), we must have
have that \(\mathcal L^m\left( \{h > 0\} \right) = 0\) and so is measurable. Since \(h \geq 0\), we then must then have that \(\{h = 0\}\) is a measurable set. With both of these facts, we see
that \(h\) is measurable. This combined with Federer's argument proves condition one in the case that \(g\) is a characteristic function of measurable subset of \(f(A)\).

For more general measurable \(g\) a standard approximation argument is used much like in Federer. First, for non-negative measurable \(g\), we can write \(g(y) = \sum_i r_i g_i(y)\), where
each \(g_i\) is the characteristic function of an \(\mathcal H^m\) measurable subset \(T\) of \(f(A)\) (recall that by hypothesis of the theorem that \(g\) is defined everywhere).
We have a corresponding decomposition of integrands \(h = \sum_i r_i h_i\) for every \(x \in A\). By our work above, we know that each \(h_i\) is measurable and that the result holds again
for \(g\). For measurable \(g\) of any sign, we do the usual for \(g = g^+ - g^-\). 

Note that the theorem requires that \(g\) is a measrable function defined everywhere, but usually we only require that measurable functions be defined up to a set with measure zero. If \(g\)
is undefined on a set \(U\) such that \(\mathcal H^m(U) = 0,\) then it is possible that \(\mathcal L^m\left(f^{-1}(U)\right) > 0\). However, from the discussion above we see that
\(\mathcal L \left(f^{-1}(U) \cap \{J_m f > 0\}\right) = 0\). Hence the integral \(\int f(g(x)) J_m f(x) \mathcal L^m\) will be independent of any values we choose to ``fill in" for \(g\) on \(U\). So
in this sense the result will still hold for \(g\) defined up to sets with measure zero.



\end{document}
