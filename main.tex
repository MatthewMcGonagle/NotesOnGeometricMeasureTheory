\documentclass{article}
\usepackage{mathtools}
\usepackage{amsthm}

\newtheorem{lemma}{Lemma}
\newtheorem{proposition}{Proposition}

\begin{document}
\title{Notes on Geometric Measure Theory}
\author{Matthew McGonagle}
\maketitle
\section{Introduction}
The purpose of these notes is mostly to expand discussions or explain details of arguments
in Federer's Geometric Measure Theory. So the layout of these notes will reflect the relevant location
of the topic from Federer.

Note that we also use Federer's definition of a measure, which in standard treatments would be called
an outer measure.

\section{Measure Theory (Federer Chapter 2)}
\subsection{Suslin Sets}

\subsubsection{A Compact Subspace of \(\mathcal N\) (Federer 2.2.12)}

Given a sequence of positive integers \(m_i\), consider the sub-space
\(K \coloneqq \{n \in \mathcal N \mid n_i \leq m_i\} \subset \mathcal N\). We show that \(K\) is a compact
sub-space of \(\mathcal N\). 

\subsection{An Identity for the Upper Integral based on Caratheodory's Construction (Federer 2.10.24 (1))} 
Federer shows that the quantity \(\lambda_\delta(f)\) based on Caratheodory's construction matches the upper
Lebesgue Integral \(\int^* f d\psi\) in certain cases. We will prove the following
statement from the case of (Federer 2.10.24 (1)),
\begin{proposition}
In the case of (Federer 2.10.24 (1)), there exists an increasing sequence of
Borel sets \(B_i\) and a sequence of \(\epsilon_i > 0\) such that
\begin{equation}
\psi\left(\{x \mid f(x) > 0 \} \setminus \bigcup_i B_i \right) = 0,
\end{equation}
and \(\psi(B_i \cap S) \leq (1 + i^{-1}) \zeta(S)\) whenever \(S \in F\), \(B_i \cap S \neq \emptyset\),
and \(\text{diam }S \leq \epsilon_i\).
\end{proposition}

Now we prove the proposition.
\begin{proof}
It is clear that we may write \(\{x \mid f(x) > 0\} = \bigcup_j A_j\) for an increasing sequence of sets
\(A_j\) with \(\psi(A_j) < \infty\). Given \(i\),  consider the set of
\(C_{ij} = \{x \mid \psi(A_i \cap S) \leq (1 + i^{-1}) \zeta(S) \text{ for } x \in S
    \text{ and diam } S \leq 1/j\}\). Note that \(C_{ij}\) is increasing in \(j\) and that 
\(\psi\left( A_i \setminus \bigcup_{j} C_{ij} \right) = 0\) by (Federer 2.10.18 (3)). Therefore, we may
find a set \(\tilde C_i\) and \(\epsilon_i > 0\) such that for all \(x \in \tilde C_i\) we have that
\(\psi(A_i \cap S) \leq (1 + i^{-1}) \zeta(S)\)
whenever \(x \in S\) and \(\text{diam } S < \epsilon_i\)
; furthermore \(\psi(A_i \setminus \tilde C_i) \leq 2^{-i - 1}\). In particular,
\(\psi(\tilde C_i \cap S) \leq (1 + i^{-1}) \zeta(S)\) whenever \(S \cap \tilde C_i \neq \emptyset\) and
\(\text{diam } S \leq \epsilon_i\).

From the Borel regularity of \(\psi\) we may find Borel sets \(B_i\) such that \(\tilde C_i \subset B_i\) and
\(\psi(B_i) = \psi(\tilde C_i)\); note that \(B_i\) must then be a \(\psi\)-hull of \(\tilde C_i\).
In particular, since
each \(S \in F\) is closed and hence \(\psi\) measurable, we have that 
\(\psi(B_i \cap S) \leq (1 + i^{-1}) \zeta(S)\) whenever \(S \cap \tilde C_i \neq \emptyset\) and
\(\text{ diam } S \leq \epsilon_i\). Since each \(S \in F\) is measurable (since closed) and
\(B_i\) is a \(\psi\)-hull of \(\tilde C_i\), we have that \(\psi(B_i \cap S) = \psi(\tilde C_i \cap S)\) for all
\(S \in F\). Furthermore, if \(S \cap B_i \neq \emptyset\) but
\(S \cap \tilde C_i = \emptyset\), then \(\psi(B_i \cap S) = 0\). So we we have
that \(\psi(B_i \cap S) \leq (1 + i^{-1}) \zeta(S)\) whenever \(S \cap B_i \neq \emptyset\) and
\(\text{ diam } S \leq \epsilon_i\).

To form an increasing sequence of Borel sets, we set \(\tilde B_i = \bigcap_{j \geq i} B_j\). We see that
\(\psi(A_i \setminus \tilde B_i) \leq \sum_{j \geq i} \psi(A_i \setminus B_j) \leq 2^{-i}\). Furthermore,
since \(\tilde B_i \subset B_i\), we have that \(\psi(\tilde B_i \cap S) \leq (1 + i^{-1}) \zeta(S)\) whenever
\(S \cap \tilde B_i \neq \emptyset\) and
\(\text{ diam } S \leq \epsilon_i\).

Next, note that \(\psi\left(A_i \setminus \bigcup_j \tilde B_j\right) \leq \psi(A_i \setminus \tilde B_j) \leq
2^{-j}\)
for any \(j \geq i\) (recall that \(A_i\) is increasing). Therefore,
\(\psi\left(A_i \setminus \bigcup_j \tilde B_j\right) = 0\), and we get that
\(\psi\left( \{ x \mid f(x) > 0 \} \setminus \bigcup_j \tilde B_j \right) = 0\).
\end{proof}


\end{document}
