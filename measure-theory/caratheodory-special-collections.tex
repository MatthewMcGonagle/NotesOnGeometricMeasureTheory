\subsection{Collections of Closed Intervals that are Special for a Given Interval (Federer 2.10.28)}

Here we discuss in more detail the proof of item (3) of Federer 2.10.28. In particular, we show that 
\(\psi(A \cap T) = \zeta(T)\) when \(T \in \bigcup_j H_j\).

\begin{lemma}
Consider a covering of \(A \cap T\) by compact intervals \(R_i \in F\). Given \(\epsilon > 0\), we may
cover \(A \cap T\) by a countable collection \(S_i \in \bigcup_j H_j\) such that each \(S_i \subset T\) and
\(\sum_i \zeta(S_i) \leq \epsilon + \sum_i \zeta(R_i)\).
\end{lemma}
\begin{proof}
Consider any compact interval \(R\). Given \(j\), consider the
\(S \in H_j\) such that \(S \subset T\) and \(S \cap R \neq \emptyset\). Since \(R\) is a compact interval
and the \(S\) are compact disjoint intervals, there are at most
two different \(S \in H_j\) such that \(S \cap R \neq \emptyset\) and \(S\) is not a subset of \(R\); these
are any \(S\) that contain the endpoints of \(R\) but are not subsets of \(R\). 

Since, \(\lim\limits_{j\to \infty}\sup\limits_{S\in H_j} \text{diam } S = 0\), given \(\epsilon > 0\) and
\(k\) we may find \(j\) and \(S_i \in H_j\) satisfying \(S_i \subset T\),
\(T \cap A \cap R \subset \bigcup_i S_i\), and 
\(\sum_i \zeta(S_i) \leq \epsilon 2^{-1-k} + \zeta(R)\).

Now given a collection of compact intervals \(R_i\) satisfying \(A \cap T \subset \bigcup_i R_i\), do the above
to find \(S_{ij} \subset T\) such that \(A \cap T \cap R_i \subset \bigcup_j S_{ij}\) and
\(\sum_j \zeta(S_{ij}) \leq \epsilon 2^{-1 - i} + \zeta(R_i)\). Then we find that the countable collection
\(\bigcup_i \bigcup_j \{S_{ij}\}\) satsifies
\(\sum_i \zeta(S_i) \leq \epsilon + \sum_i \zeta(R_i)\). 
\end{proof}

From the above lemma, when considering \(\psi(A\cap T)\), we need only consider covering \(A \cap T\) by
countable collections of \(S_i \subset T\).

\begin{lemma}
Given a countable collection \(\mathcal C\) of \(S_i \subset T\) covering \(T \cap A\), there is a
finite sub-covering of
\(T \cap A\).
\end{lemma}
\begin{proof}
We will be taking advantage of the nested nature of the \(H_j\). First define the compact sets
\(C_j = \bigcup \{S \mid S \in H_j, S \subset T, \text{ and } S \text{ not contained in some } R\in\mathcal C\}\).
Notice that 
\(\bigcap_j C_j \subset T \cap A\) and any point in this intersection is not covered by \(\mathcal C\);
therefore \(\bigcap_j C_j = \emptyset\). Furthermore, since \(C_j\) for a decreasing sequence of compact sets,
we must have that \(C_j = \emptyset\) for some \(j\). Therefore, every \(S\in H_j\) such that \(S\subset T\) is
contained in some \(R\in\mathcal C\). Since the \(S \in H_j\) such that \(S\subset T\) cover \(T\cap A\), we have
that the set of such \(R\) also cover \(T\cap A\). Finally, since \(H_j\) is a finite collection of sets, we have
a finite sub-collection of \(mathcal C\) that covers \(T \cap A\). 
\end{proof}

Next, note that since any \(S \in H_j\) satisfies \(\zeta(S) = \sum\limits_{R\in H_j, R \subset S} \zeta(R)\), for
any finite cover of \(T\cap A\) by \(S \in \bigcup_j H_j\), we find that for some \(k\)
\(\sum_i\zeta(S_i) = \sum\limits_{R \in H_k, R\subset T} \zeta(R) = \zeta(T)\). Furthermore, the \(R\in H_k\)
such that \(R \subset T\) also cover \(T \cap A\), and so we must have that \(\phi_\delta(T\cap A) \geq \zeta(T)\). 
Using the properties of speciality, we clearly have the opposite 
inequality. Therefore \(\phi_\delta(T\cap A) = \zeta(T)\), and we have that \(\psi(T\cap A) = \zeta(T)\).
