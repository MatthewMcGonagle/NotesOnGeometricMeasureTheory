\subsection{An Identity for the Upper Integral based on Caratheodory's Construction (Federer 2.10.24 (1))} 
Federer shows that the quantity \(\lambda_\delta(f)\) based on Caratheodory's construction matches the upper
Lebesgue Integral \(\int^* f d\psi\) in certain cases. For the case  present in (Federer 2.10.24 (1)), he
states that one can use (Federer 2.10.18 (3)) to show that there exists an increasing sequence of
Borel sets \(B_i\) and a sequence of \(\epsilon_i > 0\) such that
\begin{equation}
\psi\left(\{x \mid f(x) > 0 \} \setminus \bigcup_i B_i \right) = 0,
\end{equation}
and \(\psi(B_i \cap S) \leq (1 + i^{-1}) \zeta(S)\) whenever \(S \in F\), \(B_i \cap S \neq \emptyset\),
and \(\text{diam }S \leq \epsilon_i\). Here we will demonstrate this statement.

\begin{proof}
By assumption, we have that \(X\) is the union of a countable collection of sets \(A_i\) with
\(\psi(A_i) < \infty\). Furthermore, it is clear that we may take these \(A_i\) to be increasing.
From the Borel regularity of \(\psi\) we may also take the sets \(A_i\) to be an increasing Borel; that is,
the Borel regularity gives us an \(\tilde A_i\) that is Borel such that \(A_i \subset \tilde A_i\) and
\(\psi(A_i) = \psi(\tilde A_i)\). Then we can set \(A_i'  = \bigcap\limits_{j \geq i} \tilde A_j\), and we
see that \(A_i'\) is an increasing sequence of Borel sets with finite measure and \(\psi(A_i') = \psi(A_i)\). 
Therefore, we may originally consider the \(A_i\) to be an increasing sequence of Borel sets.
\end{proof}
