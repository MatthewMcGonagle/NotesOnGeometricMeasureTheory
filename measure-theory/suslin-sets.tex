\subsection{Suslin Sets}

Let \(\mathcal P = \{1, 2, 3, ...\}\) denote the set of positive integers and let
\(\mathcal N = {\mathcal P}^{\mathcal P}\) be the set of sequences of positive integers.

The topology on \(\mathcal N\) is generated by the following open sets: for any \(x \in \mathcal N\) and
\(i_0 \geq 1\), the open set
\(U = \{(x_1, ..., x_{i_0 - 1}, y_{i_0}, y_{i_0 + 1}, ...) \mid y_i \in \mathcal P\}\) for some \(i_0 \geq 1\).
This coincides with the topology of the metric used by Federer.

\subsubsection{Why look at Suslin Sets? (Federer 2.2.10)}

A Suslin set is defined to be the image of the projection \(p: X \times \mathcal N \to X\) for some closed subset \(C\) of \(X \times \mathcal N\). 
Why do we look at Suslin sets? They are subsets of \(X\) that may be complicated but can be realized as the ``shadow" of a more simple set (i.e. a closed set). 
The choice of using \(\mathcal N\) as a vertical comes from the fact that some of its topological properties allow us to show that the Suslin sets are closed
under countable union and countable intersection (see Federer's discussion of why every Borel set is a Suslin set). Since every Borel set is a Suslin set,
then we can see that the more ``complicated" notion of Borel set can be seen as a ``shadow" of a more simple closed subset \(X \times \mathcal N\).

\subsubsection{A Compact Subspace of \(\mathcal N\) (Federer 2.2.12)}

Given a sequence of positive integers \(m_i\), consider the sub-space
\(K \coloneqq \{n \in \mathcal N \mid n_i \leq m_i\} \subset \mathcal N\). We show that \(K\) is a compact
sub-space of \(\mathcal N\). First note that if \(m_i > 1\) for only a finitely many \(i\), then \(K\) is a finite
set and must be compact. Therefore, we need only consider the case that \(m_i > 1\) for infinitely many  \(i\).

To accomplish this, we use the lexiographic ordering on \(\mathcal N\). That is, \(x < y\) for
\(x, y \in \mathcal N\) if \(x_i < y_i\) for the smallest \(i\) such that \(x_i \neq y_i\).

First, a lemma. 
\begin{lemma}
Given bounded set \(S \subset \mathcal N\), we have that \(\sup S\) exists, i.e. that \(S\) has least
upper bound.
\end{lemma}
\begin{proof}
We recursively construct \(m \in \mathcal N\) such that \(s \leq m\) for all \(s \in S\). Since, \(S\) is
bounded, we have that the set of positive integers \(\{s_1 \mid s \in S\}\) is bounded. We let \(m_1\) be this
maximum. 

Now, given \(j\), suppose we have constructed \(m_i\) for \(i < j\) such that there exists \(s \in S\) with
\(s_i = m_i\) for \(i < j\) and for any \(s \in S\) we have that \(s_i \leq m_i\) for \(i < j\). 

Now, if the set of positive integers
\(S_j \coloneqq \{s_j \mid s \in S \text{ and } s_i = m_i \text{ for } i < j\}\) is unbounded, then we may choose 
\(m = (m_1,..., m_{j-2}, 1 + m_{j-1}, 1, 1, ...)\) and we see that \(m\) is a least upper bound of \(S\).
So consider the case that this set is bounded. Then we let \(m_j\) be its maximum.

We have recursively constructed our \(m_j\). We need only consider the case that the sets \(S_j\) are bounded
for each step of our recursion. Clearly \(m\) is an upper bound for \(S\). Furthermore, consider any \(x < m\)
such that \(i\) is the first component where \(x_i < m_i\). Then by construction, there is an \(s \in S_i\) such
that \(s > x\). Hence \(m\) is a least upper bound. 
\end{proof}

Now consider an open covering \(U_\alpha\) of \(K\) and the set \(S\) of points \(s \in \mathcal N\) such
that \(s \leq m\) and the set \(K_s \coloneqq \{k \mid k \in K \text{ and } k \leq s\}\) is covered by a
finite sub-collection of \(U_\alpha\). Since \(1 \coloneqq (1, 1, 1, ...) = \min K\), we see that \(1 \in S\).
Furthermore, by construction \(S\) is bounded. 

Next, note that for some \(i_0\), all points in \(K\) of the form \((1, 1, ..., 1, n_{i_0}, n_{i_0 + 1}, ...)\)
are in an open set \(U_1\) covering \(1\). Since \(m_i > 1\) for infinitely many \(i\), this means that there
exists \(k_0 \in K \cap U_1\) such that all \(K_{k_0} \subset U_1\). Therefore \(M \coloneqq \sup S > 1\).

Given any \(x \in \mathcal N\) that is an upper bound for \(S\), we have that \(y \in \mathcal N\) defined by
\(y_i \coloneqq \min \{x_i, m_i\}\) is also an upper bound for \(S\). Therefore \(M_i \leq m_i\); note that
this is a stronger condition than \(M \leq m\). 

We wish to show that \(M \in S\). Since \(M_i \leq m_i\), we have that \(M \in K\). So let \(U_M\) be an
open set covering \(M\). So we know that there exists \(i_0\) such that all
points in \(K\) of the form \((M_1, M_2, ..., M_{i_0}, n_{i_0+1}, n_{i_0 + 2}, ...)\) for
general \(n_i\) are in \(U_M\).  

Consider the case that \(M_{j_0} > 1\) for some \(j_0 > i_0\). Then there exists a point \(k \in K\) such that
\(k < M\) and \(k \in U_M\), namely \(k = (M_1, ..., M_{i_0}, 1, 1, 1, ...)\). Note that all \(x \in K\)
satisfying \(k \leq x \leq M\) are also in \(U_M\). By the definition of \(M\), we
have that \(K_k\) is covered by a finite sub-collection \(U_\alpha\). When combined with \(U_M\)
we see that \(K_M\) is also covered by a finite sub-collection and so \(M \in S\). 

Consider the final case that \(M_j = 1\) for all \(j > i_0\), i.e. \(M = (M_1, ..., M_{i_0}, 1, 1, 1, ...)\).
We know that \(M > 1\) so there must exist a greatest \(j_0 \leq i_0\) such that \(M_{j_0} > 1\). Consider
the point
\(x = (M_1, ..., M_{j_0}-1, m_{j_0 + 1}, m_{j_0 + 2}, ...) \in K\). We have that \(x < M\) and that there are
no points \(y \in K\) such that \(x < y < M\). We may cover \(K_x\) with a finite sub-collection of the
\(U_\alpha\), and then we combined with \(U_M\) we have a finite sub-collection covering \(K_M\). Therefore
\(M \in S\). 

We prove by contradiction that \(M = m\). Consider if \(M < m\). Similar to above, consider an open set
\(U_M\) covering \(M\), and consider the case that there is a \(j_0 > i_0\) such that \(M_{j_0} < m_{j_0}\).
Let \(x = (M_1, ..., M_{j_0 - 1}, m_{j_0}, 1, 1, 1, ...)\); note that all
\(y \in K\) such that \(M < y \leq x\) are also covered by \(U_M\). Therefore \(K_x\) will also be
covered by a finite sub-collection of \(U_\alpha\), which contradicts the definition of \(M\).

Consider the final case that \(M_i = m_i\) for all \(i > i_0\) but that there exists a \(j_0 \leq i_0\) with
\(M_{j_0} < m_{j_0}\). We may take \(j_0\) to be the largest such number. Let 
\(x = (M_1, ..., M_{j_0 - 1}, M_{j_0} + 1, 1, 1, 1, ...) \in K\). We note that there are no \(y \in K\) such
that \(M < y < x\). Since \(M \in S\), we may cover \(K_M\) with a finite sub-collection of \(U_\alpha\); this
combined with a cover \(U_x\) of \(x\) gives a finite sub-cover for \(K_x\), which contradicts the
definition of \(M\). 

Therefore we have \(M = m\) and \(K\) may be covered by a finite sub-collection of \(U_\alpha\).
