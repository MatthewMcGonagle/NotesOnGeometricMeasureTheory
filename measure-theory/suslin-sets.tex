\subsection{Suslin Sets}

\subsubsection{A Compact Subspace of \(\mathcal N\) (Federer 2.2.12)}

Given a sequence of positive integers \(m_i\), consider the sub-space
\(K \coloneqq \{n \in \mathcal N \mid n_i \leq m_i\} \subset \mathcal N\). We show that \(K\) is a compact
sub-space of \(\mathcal N\).

To accomplish this, we use the lexiographic ordering on \(\mathcal N\). That is, \(x < y\) for
\(x, y \in \mathcal N\) if \(x_i < y_i\) for the smallest \(i\) such that \(x_i \neq y_i\).

First, a lemma. 
\begin{lemma}
Given bounded set \(S \subset \mathcal N\), we have that \(\sup S\) exists, i.e. that \(S\) has least
upper bound.
\end{lemma}
\begin{proof}
We recursively construct \(m \in \mathcal N\) such that \(s \leq m\) for all \(s \in S\). Since, \(S\) is
bounded, we have that the set of positive integers \(\{s_1 \mid s \in S\}\) is bounded. We let \(m_1\) be this
maximum. 

Now, given \(j\), suppose we have constructed \(m_i\) for \(i < j\) such that there exists \(s \in S\) with
\(s_i = m_i\) for \(i < j\) and for any \(s \in S\) we have that \(s_i \leq m_i\) for \(i < j\). 

Now, if the set of positive integers
\(S_j \coloneqq \{s_j \mid s \in S \text{ and } s_i = m_i \text{ for } i < j\}\) is unbounded, then we may choose 
\(m = (m_1,..., m_{j-2}, 1 + m_{j-1}, 1, 1, ...)\) and we see that \(m\) is a least upper bound of \(S\).
So consider the case that this set is bounded. Then we let \(m_j\) be its maximum.

We have recursively constructed our \(m_j\). We need only consider the case that the sets \(S_j\) are bounded
for each step of our recursion. Clearly \(m\) is an upper bound for \(S\). Furthermore, consider any \(x < m\)
such that \(i\) is the first component where \(x_i < m_i\). Then by construction, there is an \(s \in S_i\) such
that \(s > x\). Hence \(m\) is a least upper bound. 
\end{proof}

Now consider the set \(S\) of points \(s \in \mathcal N\) such that the set
\(K_s \coloneqq \{k \mid k \in K \text{ and } k \leq s\}\) is compact. Since
\(1 \coloneqq (1, 1, 1, ...) = \min K\), we see that \(1 \in S\). Furthermore, \(\max K = m\); so \(S\) is
bounded.

Let \(M = \sup S\). We wish to show that \(M \in S\). So let \(K_M\) be covered by a collection of open sets
\(U_\alpha\). Let \(U_0\) be an open set covering \(M\). So we know that there exists \(i_0\) such that all
points in \(K\) of the form \((M_1, M_2, ..., M_{i_0}, n_{i_0+1}, n_{i_0 + 2}, ...)\) for
general \(n_i\) are in \(U_0\).  

Consider the case that \(M_i > 1\) for some \(i > i_0\). Then there exists a point \(k \in K\) such that
\(k < M\) and \(k \in U_0\), namely \(k = (M_1, ..., M_{i_0}, 1, 1, 1, ...)\). By the definition of \(M\), we
have that \(K_k\) is compact. So we cover \(K_M\) in a finite number of \(U_alpha\) and \(K_M\) is compact.

Consider the final case that \(M_i = 1\) for all \(i > i_0\), i.e. \(M = (M_1, ..., M_{i_0}, 1, 1, 1, ...)\).
From the definition of \(K\) we know that \(k = (M_1, ..., M_{i_0} - 1, m_{i_0 + 1}, m_{i_0 + 2}, ...) \in K\),
\(k < M\), and there are no \(x \in K\) such that \(k < x < M\). TODO: Need to handle \(M_{i_0} = 1\).
